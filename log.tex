% build with xelatex and miktex
\documentclass[UTF8, 10pt, a4paper]{article}
\usepackage[UTF8]{ctex}
\usepackage[margin=0.7in]{geometry}
\usepackage{authblk}
\usepackage{siunitx}
\usepackage{amsmath}
\usepackage{graphicx}
\usepackage{subfigure}
\usepackage{hyperref}
\title{KAUS学习科研日志}
\author{石永祥,Shi Yongxiang\footnote{Peking University, Email: shiyongxiang@pku.edu.cn}}

\begin{document}
    \maketitle
    \begin{abstract}
        这是我在KAUST期间的相关工作,跟着Gerald Schuster。主要是初步接触FWI与相关的工作的时候,觉得整理一下方便之后的查看。主要包括几个大的部分, FWI,RTM与相关的实现方法,如有限差分(Finite Difference)等相关方面。记录的顺序大致是日期顺序。
    \end{abstract}
    
    
    
\end{document}
